% Copyright (C) 2014-2019 by Thomas Auzinger <thomas@auzinger.name>

\documentclass[draft,final]{vutinfth} % Remove option 'final' to obtain debug information.

% Load packages to allow in- and output of non-ASCII characters.
\usepackage{lmodern}        % Use an extension of the original Computer Modern font to minimize the use of bitmapped letters.
\usepackage[T1]{fontenc}    % Determines font encoding of the output. Font packages have to be included before this line.
\usepackage[utf8]{inputenc} % Determines encoding of the input. All input files have to use UTF8 encoding.

% Extended LaTeX functionality is enables by including packages with \usepackage{...}.
\usepackage{amsmath}    % Extended typesetting of mathematical expression.
\usepackage{amssymb}    % Provides a multitude of mathematical symbols.
\usepackage{mathtools}  % Further extensions of mathematical typesetting.
\usepackage{microtype}  % Small-scale typographic enhancements.
\usepackage[inline]{enumitem} % User control over the layout of lists (itemize, enumerate, description).
\usepackage{multirow}   % Allows table elements to span several rows.
\usepackage{booktabs}   % Improves the typesettings of tables.
\usepackage{subcaption} % Allows the use of subfigures and enables their referencing.
\usepackage[ruled,linesnumbered,algochapter]{algorithm2e} % Enables the writing of pseudo code.
\usepackage[usenames,dvipsnames,table]{xcolor} % Allows the definition and use of colors. This package has to be included before tikz.
\usepackage{nag}       % Issues warnings when best practices in writing LaTeX documents are violated.
\usepackage{todonotes}
\usepackage{hyperref}  % Enables cross linking in the electronic document version. This package has to be included second to last.
\usepackage[acronym,toc]{glossaries} % Enables the generation of glossaries and lists fo acronyms. This package has to be included last.

\usepackage{textcomp} % provide euro and other symbols
% Use upquote if available, for straight quotes in verbatim environments
\IfFileExists{upquote.sty}{\usepackage{upquote}}{}
\IfFileExists{microtype.sty}{% use microtype if available
  \usepackage[]{microtype}
  \UseMicrotypeSet[protrusion]{basicmath} % disable protrusion for tt fonts
}{}
\makeatletter
\@ifundefined{KOMAClassName}{% if non-KOMA class
  \IfFileExists{parskip.sty}{%
    \usepackage{parskip}
  }{% else
    \setlength{\parindent}{0pt}
    \setlength{\parskip}{6pt plus 2pt minus 1pt}}
}{% if KOMA class
  \KOMAoptions{parskip=half}}
\makeatother
\usepackage{xcolor}
\IfFileExists{xurl.sty}{\usepackage{xurl}}{} % add URL line breaks if available
\IfFileExists{bookmark.sty}{\usepackage{bookmark}}{\usepackage{hyperref}}
\urlstyle{same} % disable monospaced font for URLs
\makeatletter
\makeatother
\setlength{\emergencystretch}{3em} % prevent overfull lines
\providecommand{\tightlist}{%
  \setlength{\itemsep}{0pt}\setlength{\parskip}{0pt}}
\setcounter{secnumdepth}{-\maxdimen} % remove section numbering
\usepackage{longtable}

% Define convenience functions to use the author name and the thesis title in the PDF document properties.
\newcommand{\authorname}{Johannes Altmanninger} % The author name without titles.
\usepackage{catchfile}
\CatchFileDef{\thesistitle}{../title.tex}{} % The title of the thesis. The English version should be used, if it exists.

% Set PDF document properties
\hypersetup{
    pdfpagelayout   = TwoPageRight,           % How the document is shown in PDF viewers (optional).
    % linkbordercolor = {Melon},                % The color of the borders of boxes around crosslinks (optional).
    pdfauthor       = {\authorname},          % The author's name in the document properties (optional).
    pdftitle        = {\thesistitle},         % The document's title in the document properties (optional).
    pdfsubject      = {},              % The document's subject in the document properties (optional).
    pdfkeywords     = {} % The document's keywords in the document properties (optional).
    hidelinks,
    colorlinks,
    linkcolor={red!50!black},
    citecolor={blue!50!black},
    urlcolor={blue!80!black}
}

\setpnumwidth{2.5em}        % Avoid overfull hboxes in the table of contents (see memoir manual).
\setsecnumdepth{subsection} % Enumerate subsections.

\nonzeroparskip             % Create space between paragraphs (optional).
\setlength{\parindent}{0pt} % Remove paragraph identation (optional).

\makeindex      % Use an optional index.
\makeglossaries % Use an optional glossary.
%\glstocfalse   % Remove the glossaries from the table of contents.

% Set persons with 4 arguments:
%  {title before name}{name}{title after name}{gender}
%  where both titles are optional (i.e. can be given as empty brackets {}).
\setauthor{}{\authorname}{}{male}
\setadvisor{Associate Prof. Dipl.-Ing.}{Georg Weissenbacher}{D.Phil.}{male}

\setfirstassistant{}{Adrián Rebola-Pardo}{MSc.}{male}

% For dissertations:
\setfirstreviewer{Pretitle}{Forename Surname}{Posttitle}{male}
\setsecondreviewer{Pretitle}{Forename Surname}{Posttitle}{male}

% For dissertations at the PhD School and optionally for dissertations:
\setsecondadvisor{Pretitle}{Forename Surname}{Posttitle}{male} % Comment to remove.

% Required data.
\setregnumber{01455764}
\setdate{01}{10}{2019} % Set date with 3 arguments: {day}{month}{year}.
\settitle{\thesistitle}{\thesistitle}
% Sets English and German version of the title (both can be English or German). If your title contains commas, enclose it with additional curvy brackets (i.e., {{your title}}) or define it as a macro as done with \thesistitle.
\setsubtitle{}{} % Sets English and German version of the subtitle (both can be English or German).

% Select the thesis type: bachelor / master / doctor / phd-school.
% Bachelor:
\setthesis{master}
%
% Master:
%\setthesis{master}
\setmasterdegree{master} % dipl. / rer.nat. / rer.soc.oec. / master
%
% Doctor:
%\setthesis{doctor}
%\setdoctordegree{rer.soc.oec.}% rer.nat. / techn. / rer.soc.oec.
%
% Doctor at the PhD School
%\setthesis{phd-school} % Deactivate non-English title pages (see below)

% For bachelor and master:
\newcommand{\emcl}{European Master's Program in Computational Logic}
\setcurriculum{\emcl}{\emcl} % Sets the English and German name of the curriculum.

% For dissertations at the PhD School:
\setfirstreviewerdata{Affiliation, Country}
\setsecondreviewerdata{Affiliation, Country}


\begin{document}

\frontmatter % Switches to roman numbering.
% The structure of the thesis has to conform to
%  http://www.informatik.tuwien.ac.at/dekanat

\addtitlepage{naustrian} % German title page (not for dissertations at the PhD School).
\addtitlepage{english} % English title page.
\addstatementpage

\if0
\begin{danksagung*}
    TODO
\end{danksagung*}

\begin{acknowledgements*}
    TODO
\todo{Enter your text here.}
\end{acknowledgements*}
\fi

\begin{kurzfassung}

Das klauselbasierte Beweisformat DRAT ist der De-Facto-Standard, um
Unerfüllbarkeitsergebnisse von SAT-Gleichungslösern zu verifizieren.
DRAT-Beweise bestehen aus Lemmas, die Klauseln hinzufügen,
und Klauseleliminierungen.  Moderne DRAT-Beweisprüfer ignorieren
Eliminierungsinstruktionen von sogenannten Unit-Klauseln, was bedeutet,
dass sich die Semantik der Beweisprüfer von der DRAT-Spezifikation
unterscheidet; infolgedessen können sie manche, von SAT-Lösern verwendete
Vereinfachungstechniken, die Unit-Klauseln eliminieren unter Umständen
nicht verifizieren.  Moderne SAT-Löser generieren Beweise die von diesen
DRAT-Prüfern akzeptiert werden, jedoch bezüglich der DRAT-Spezifikation
inkorrekt sind, da sie Eliminierungsinstruktionen enthalten, die
nicht-redundante Informationen löschen.  Wir schlagen Korrekturen für
prämierte SAT-Löser vor, wodurch diese in der Lage sind, Beweise ohne
ebenjene kontraproduktiven Eliminierungen zu generieren, die ergo im Sinne der
Spezifikation korrekt sind. Dennoch können Unit-Eliminierungen in Beweisen
in Anbetracht der Verwendung von fortgeschrittenen Vereinfachungstechniken
wohl kaum ausgeschlossen werden, sofern der Löseaufwand nicht darunter leiden
soll.  Die Durchführung von Unit-Eliminierungen in Beweisprüfern kann viel
Rechenzeit beanspruchen.  Wir haben den ersten wettbewerbsfähigen Prüfer
implementiert, der Unit-Eliminierungen berücksichtigt and präsentieren
unsere Versuchsergebnisse, die darauf hindeuten, dass der Rechenaufwand
für das Überprüfen eines durchschnittlichen Beweises mit oder ohne
Unit-Eliminierungen gleich ist.  Weiters führen wir das SICK-Format ein,
das, vergleichsweise kleine und effizient überprüfbare Gegenbeispiele
zu DRAT-Beweisen beschreibt. Indem wir diese Gegenbeispiele mit einem
unabhängigen Programm überprüfen, stärken wir das Vertrauen in die
Stichhaltigkeit der Inkorrektheits-Ergebnisse.  Außerdem kann diese Technik
von Nutzen sein um Fehler in (der Beweisgenerierung von) SAT-Lösern und
Prüfern zu finden.

\end{kurzfassung}

\begin{abstract}
\input{../abstract}
\end{abstract}

% Select the language of the thesis, e.g., english or naustrian.
\selectlanguage{english}

% Add a table of contents (toc).
\tableofcontents % Starred version, i.e., \tableofcontents*, removes the self-entry.

% Switch to arabic numbering and start the enumeration of chapters in the table of content.
\mainmatter

\input{thesis-for-submission.tex}

\backmatter

% Use an optional list of figures.
% \listoffigures % Starred version, i.e., \listoffigures*, removes the toc entry.

% Use an optional list of tables.
% \cleardoublepage % Start list of tables on the next empty right hand page.
%\listoftables % Starred version, i.e., \listoftables*, removes the toc entry.

% Use an optional list of alogrithms.
%\listofalgorithms
%\addcontentsline{toc}{chapter}{List of Algorithms}

% Add an index.
\printindex

% Add a glossary.
\printglossaries

% Add a bibliography.
\bibliographystyle{alpha}
\bibliography{../references.bib}

\end{document}
